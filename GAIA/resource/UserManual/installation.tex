
\chapter{Installing GAIA}
\section{Getting GAIA}
\subsection{Viewing the Source Code}
One way to explore the GAIA source code is to go to
http://linqs.cs.umd.edu/trac/gaia and click on Browse Source.
This is a web interface allowing you to view the source code without
downloading anything.  The GAIA project uses SVN as its
revision management system.  Thus, the repository is laid
out using standard SVN conventions where ``trunk'' contains
the main revision of the code, ``branch'' contains different branches of the code,
and tags contains tagged points of the code representing stable revisions.

\subsection{Downloading GAIA}
The simplest way to obtain GAIA is to download it as a jar file
from http://linqs.cs.umd.edu/trac/gaia.  There are two jar files.
The first, gaia.jar, contains the binary for the GAIA.
The second file, gaia-src.jar, contains the binary, as
well as the source code for GAIA.  Both jar files were generated
using the Eclipse Export feature.  Thus, the jar file with the
source code can be directly loaded into Eclipse.

Note, some features of eclipse may require third party libraries.
See \secref{thirdparty} for what those are and where to download them.

\subsection{Checking out GAIA}
Another way to acquire GAIA is to check it out directly from the SVN.
GAIA can currently be checked out using the command:

svn co http://linqs.cs.umd.edu/svn/gaia

Note that by default, access to the SVN is read only so any modification
to the code cannot be checked in directly.  For information
on contributing your code to the project, see \secref{contributing}.

\section{Third Party Libraries}
\label{thirdparty}
Some features of GAIA require third party libraries which are not
including in the GAIA jar file and must be downloaded separately.
Links to the specific versions of the libraries are available at
http://linqs.cs.umd.edu/trac/gaia/wiki/ThirdPartyLibraries.

\section{Contributing to GAIA}
\label{contributing}
Contributions to the GAIA code base is encouraged and welcome.
Look at \chapref{advanced} for details about the GAIA code base
and conventions used in the code.  Contributions can be sent
via email to gaia-devel@cs.umd.edu.  Please include documentation
about the changes and features you'd like to contribute and
a jar file of the changes.

